\documentclass[a4paper,11pt]{article}
\usepackage{zh_CN-Adobefonts_external} % Simplified Chinese Support using external fonts (./fonts/zh_CN-Adobe/)
\usepackage{fancyhdr}  % 页眉页脚
\usepackage{minted}    % 代码高亮
\usepackage[colorlinks]{hyperref}  % 目录可跳转
\setlength{\headheight}{15pt}

% 定义页眉页脚
\pagestyle{fancy}
\fancyhf{}
\fancyhead[C]{Algorithm Library by palayutm}
\lfoot{}
\cfoot{\thepage}
\rfoot{}

\author{palayutm}   
\title{Algorithm Library}

\begin{document} 
\maketitle % 封面
\newpage % 换页
\tableofcontents % 目录

\newpage
\section{java} % 一级标题
\inputminted[breaklines]{java}{java/大数二分.java} % 插入代码文件
\inputminted[breaklines]{java}{java/加减乘除等.java} % 插入代码文件
\inputminted[breaklines]{java}{java/开头.java} % 插入代码文件
\inputminted[breaklines]{java}{java/判大素数.java % 插入代码文件

\newpage
\section{python} %一级标题
\inputminted[breaklines]{python}{python/计算表达式.py} % 插入代码文件
\inputminted[breaklines]{python}{python/正则表达式.py} % 插入代码文件

\newpage
\section{动态规划} %一级标题
\inputminted[breaklines]{c++}{动态规划/数位DP.cpp} % 插入代码文件

\newpage
\section{基本操作} %一级标题
\inputminted[breaklines]{c++}{基本操作/bitset.cpp} % 插入代码文件
\inputminted[breaklines]{c++}{基本操作/二进制枚举.cpp} % 插入代码文件
\inputminted[breaklines]{c++}{基本操作/高斯消元.cpp} % 插入代码文件
\inputminted[breaklines]{c++}{基本操作/矩阵快速幂.cpp} % 插入代码文件
\inputminted[breaklines]{c++}{基本操作/状态压缩.cpp} % 插入代码文件
\subsection{离散化} % 二级标题
\inputminted[breaklines]{c++}{基本操作/离散化/一维线离散化.cpp} % 插入代码文件
\inputminted[breaklines]{c++}{基本操作/离散化/二维点离散化.cpp} % 插入代码文件

\newpage
\section{计算几何} %一级标题
\subsection{叉积} % 二级标题
\inputminted[breaklines]{c++}{计算几何/叉积/点线式.cpp} % 插入代码文件
\inputminted[breaklines]{c++}{计算几何/叉积/重载版.cpp} % 插入代码文件
\inputminted[breaklines]{c++}{计算几何/叉积/叉积算多边形面积.cpp} % 插入代码文件
\subsection{极角排序} % 二级标题
\inputminted[breaklines]{c++}{计算几何/极角排序/atan2函数计算.cpp} % 插入代码文件
\inputminted[breaklines]{c++}{计算几何/极角排序/叉积计算.cpp} % 插入代码文件
\subsection{半平面交} % 二级标题
\inputminted[breaklines]{c++}{计算几何/半平面交/半平面交存点.cpp} % 插入代码文件
\inputminted[breaklines]{c++}{计算几何/半平面交/判不等式组是否有解.cpp} % 插入代码文件
\subsection{凸包} % 二级标题
\inputminted[breaklines]{c++}{计算几何/凸包/一般凸包.cpp} % 插入代码文件
\inputminted[breaklines]{c++}{计算几何/凸包/最大空凸包.cpp} % 插入代码文件


\subsection{Minimum Spanning Tree} % 二级标题
\subsubsection{Kruskal} % 三级标题
\inputminted[breaklines]{c++}{graph/kruskal.cc} % 插入代码文件
% 中文测试
\subsection{单源最短路}
\subsubsection{SPFA}
\inputminted[breaklines]{c++}{graph/spfa.cc}

\twocolumn  % 分页显示
\newpage
\section{String}
\subsection{KMP}
\inputminted[breaklines]{c++}{string/kmp.cc}

\subsection{Suffix Automaton}
\inputminted[breaklines]{c++}{string/suffix-automaton.cc}

%\newpage
%\section{Others}

\end{document}
