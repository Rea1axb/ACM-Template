\documentclass[a4paper,11pt]{article}
\usepackage{zh_CN-Adobefonts_external} % Simplified Chinese Support using external fonts (./fonts/zh_CN-Adobe/)
\usepackage{fancyhdr}  % 页眉页脚
\usepackage{minted}    % 代码高亮
\usepackage[colorlinks]{hyperref}  % 目录可跳转
\setlength{\headheight}{15pt}

% 定义页眉页脚
\pagestyle{fancy}
\fancyhf{}
\fancyhead[C]{Algorithm Library by palayutm}
\lfoot{}
\cfoot{\thepage}
\rfoot{}

\author{palayutm}   
\title{Algorithm Library}

\begin{document} 
\maketitle % 封面
\newpage % 换页
\tableofcontents % 目录

\newpage
\section{java} % 一级标题
\inputminted[breaklines]{java}{java/大数二分.java} % 插入代码文件
\inputminted[breaklines]{java}{java/加减乘除等.java} % 插入代码文件
\inputminted[breaklines]{java}{java/开头.java} % 插入代码文件
\inputminted[breaklines]{java}{java/判大素数.java % 插入代码文件

\newpage
\section{python} %一级标题
\inputminted[breaklines]{python}{python/计算表达式.py} % 插入代码文件
\inputminted[breaklines]{python}{python/正则表达式.py} % 插入代码文件

\newpage
\section{动态规划} %一级标题
\subsection{数位DP} % 二级标题
\inputminted[breaklines]{c++}{动态规划/数位DP/数位DP.cpp} % 插入代码文件

\newpage
\section{基本操作} %一级标题
\inputminted[breaklines]{c++}{基本操作/bitset.cpp} % 插入代码文件
\inputminted[breaklines]{c++}{基本操作/二进制枚举.cpp} % 插入代码文件
\inputminted[breaklines]{c++}{基本操作/高斯消元.cpp} % 插入代码文件
\inputminted[breaklines]{c++}{基本操作/矩阵快速幂.cpp} % 插入代码文件
\inputminted[breaklines]{c++}{基本操作/状态压缩.cpp} % 插入代码文件
\subsection{离散化} % 二级标题
\inputminted[breaklines]{c++}{基本操作/离散化/一维线离散化.cpp} % 插入代码文件
\inputminted[breaklines]{c++}{基本操作/离散化/二维点离散化.cpp} % 插入代码文件

\newpage
\section{计算几何} %一级标题
\subsection{叉积} % 二级标题
\inputminted[breaklines]{c++}{计算几何/叉积/点线式.cpp} % 插入代码文件
\inputminted[breaklines]{c++}{计算几何/叉积/重载版.cpp} % 插入代码文件
\inputminted[breaklines]{c++}{计算几何/叉积/叉积算多边形面积.cpp} % 插入代码文件
\subsection{极角排序} % 二级标题
\inputminted[breaklines]{c++}{计算几何/极角排序/atan2函数计算.cpp} % 插入代码文件
\inputminted[breaklines]{c++}{计算几何/极角排序/叉积计算.cpp} % 插入代码文件
\subsection{半平面交} % 二级标题
\inputminted[breaklines]{c++}{计算几何/半平面交/半平面交存点.cpp} % 插入代码文件
\inputminted[breaklines]{c++}{计算几何/半平面交/判不等式组是否有解.cpp} % 插入代码文件
\subsection{凸包} % 二级标题
\inputminted[breaklines]{c++}{计算几何/凸包/一般凸包.cpp} % 插入代码文件
\inputminted[breaklines]{c++}{计算几何/凸包/最大空凸包.cpp} % 插入代码文件

\newpage
\section{数据结构} %一级标题
\subsection{单调栈&单调队列} % 二级标题
\inputminted[breaklines]{c++}{数据结构/单调栈&单调队列/单调栈.cpp} % 插入代码文件
\inputminted[breaklines]{c++}{数据结构/单调栈&单调队列/单调队列.cpp} % 插入代码文件
\subsection{线段树} % 二级标题
\subsubsection{普通线段树} % 三级标题
\inputminted[breaklines]{c++}{数据结构/线段树/普通线段树/单点修改&区间查询.cpp} % 插入代码文件
\inputminted[breaklines]{c++}{数据结构/线段树/普通线段树/区间修改&区间查询.cpp} % 插入代码文件
\inputminted[breaklines]{c++}{数据结构/线段树/普通线段树/区间修改&区间查询_矩阵ver.cpp} % 插入代码文件
\inputminted[breaklines]{c++}{数据结构/线段树/普通线段树/区间染色.cpp} % 插入代码文件
\subsubsection{二维线段树} % 三级标题
\inputminted[breaklines]{c++}{数据结构/线段树/二维线段树/单点修改&区间查询(最值&最大值).cpp} % 插入代码文件
\subsubsection{ZKW线段树} % 三级标题
\inputminted[breaklines]{c++}{数据结构/线段树/ZKW线段树/开局.cpp} % 插入代码文件
\inputminted[breaklines]{c++}{数据结构/线段树/ZKW线段树/单点修改+区间查询.cpp} % 插入代码文件
\inputminted[breaklines]{c++}{数据结构/线段树/ZKW线段树/单点修改+区间查询最大子段和.cpp} % 插入代码文件
\inputminted[breaklines]{c++}{数据结构/线段树/ZKW线段树/区间加减+单点查询.cpp} % 插入代码文件
\inputminted[breaklines]{c++}{数据结构/线段树/ZKW线段树/区间加减+区间最值查询(lazy标记).cpp} % 插入代码文件
\subsection{树状数组} % 二级标题
\subsubsection{一维树状数组} % 三级标题
\inputminted[breaklines]{c++}{数据结构/树状数组/一维树状数组/单点修改,区间查询.cpp} % 插入代码文件
\inputminted[breaklines]{c++}{数据结构/树状数组/一维树状数组/区间修改,单点查询.cpp} % 插入代码文件
\inputminted[breaklines]{c++}{数据结构/树状数组/一维树状数组/区间修改,区间查询.cpp} % 插入代码文件
\subsubsection{二维树状数组} % 三级标题
\inputminted[breaklines]{c++}{数据结构/树状数组/二维树状数组/单点修改,区间查询.cpp} % 插入代码文件
\inputminted[breaklines]{c++}{数据结构/树状数组/二维树状数组/区间修改,单点查询.cpp} % 插入代码文件
\inputminted[breaklines]{c++}{数据结构/树状数组/二维树状数组/区间修改,区间查询.cpp} % 插入代码文件
\subsection{平衡树Treap} % 二级标题
\inputminted[breaklines]{c++}{数据结构/平衡树Treap/普通平衡树Treap.cpp} % 插入代码文件
\subsection{莫队算法} % 二级标题
\inputminted[breaklines]{c++}{数据结构/莫队算法/回滚莫队.cpp} % 插入代码文件
\inputminted[breaklines]{c++}{数据结构/莫队算法/区间查询,统计两个相同概率.cpp} % 插入代码文件
\inputminted[breaklines]{c++}{数据结构/莫队算法/统计有多少个不同的数.cpp} % 插入代码文件
\inputminted[breaklines]{c++}{数据结构/莫队算法/时间戳+统计有多少个不同的数.cpp} % 插入代码文件
\inputminted[breaklines]{c++}{数据结构/莫队算法/树状数组维护区间两数之差.cpp} % 插入代码文件

\newpage
\section{图论} %一级标题
\subsection{最短路} % 二级标题
\inputminted[breaklines]{c++}{图论/最短路/Dijkstra+堆优化.cpp} % 插入代码文件
\inputminted[breaklines]{c++}{图论/最短路/Floyd.cpp} % 插入代码文件
\inputminted[breaklines]{c++}{图论/最短路/第k短路.cpp} % 插入代码文件
\subsubsection{SPFA} % 三级标题
\inputminted[breaklines]{c++}{图论/最短路/SPFA/DFS优化.cpp} % 插入代码文件
\inputminted[breaklines]{c++}{图论/最短路/SPFA/LLL优化.cpp} % 插入代码文件
\inputminted[breaklines]{c++}{图论/最短路/SPFA/SLF+容错.cpp} % 插入代码文件
\inputminted[breaklines]{c++}{图论/最短路/SPFA/SPFA(可判环).cpp} % 插入代码文件
\subsection{最小环} % 二级标题
\inputminted[breaklines]{c++}{图论/最小环/Dijkstra+剪枝.cpp} % 插入代码文件
\inputminted[breaklines]{c++}{图论/最小环/Floyd.cpp} % 插入代码文件
\subsection{树} % 二级标题
\inputminted[breaklines]{c++}{图论/树/树的直径.cpp} % 插入代码文件
\inputminted[breaklines]{c++}{图论/树/树上倍增.cpp} % 插入代码文件
\inputminted[breaklines]{c++}{图论/树/LCA(倍增).cpp} % 插入代码文件
\inputminted[breaklines]{c++}{图论/树/树上差分.cpp} % 插入代码文件
\inputminted[breaklines]{c++}{图论/树/树链剖分.cpp} % 插入代码文件
\subsubsection{最小生成树} % 三级标题
\inputminted[breaklines]{c++}{图论/树/最小生成树/Boruvka.cpp} % 插入代码文件
\subsection{点分治} % 二级标题
\inputminted[breaklines]{c++}{图论/点分治/树上路径长度为k的点对数.cpp} % 插入代码文件
\subsection{图的连通性} % 二级标题
\subsubsection{无向图连通性} % 三级标题
\inputminted[breaklines]{c++}{图论/图的连通性/无向图连通性/边双连通分量.cpp} % 插入代码文件
\inputminted[breaklines]{c++}{图论/图的连通性/无向图连通性/点双连通分量.cpp} % 插入代码文件
\subsubsection{有向图连通性} % 三级标题
\inputminted[breaklines]{c++}{图论/图的连通性/有向图连通性/强连通分量.cpp} % 插入代码文件
\subsection{网络流} % 二级标题
\subsubsection{最大流} % 三级标题
\inputminted[breaklines]{c++}{图论/网络流/最大流/最大流.cpp} % 插入代码文件
\subsubsection{最小费用流} % 三级标题
\inputminted[breaklines]{c++}{图论/网络流/最小费用流/mcmf(augement).cpp} % 插入代码文件
\inputminted[breaklines]{c++}{图论/网络流/最小费用流/mcmf(dijkstra).cpp} % 插入代码文件
\inputminted[breaklines]{c++}{图论/网络流/最小费用流/mcmf(spfa).cpp} % 插入代码文件
\subsubsection{二分图匹配} % 三级标题
\inputminted[breaklines]{c++}{图论/网络流/二分图匹配/匈牙利算法.cpp} % 插入代码文件
\inputminted[breaklines]{c++}{图论/网络流/二分图匹配/Hopcroft-Karp算法.cpp} % 插入代码文件


\subsection{Minimum Spanning Tree} % 二级标题
\subsubsection{Kruskal} % 三级标题
\inputminted[breaklines]{c++}{graph/kruskal.cc} % 插入代码文件
% 中文测试
\subsection{单源最短路}
\subsubsection{SPFA}
\inputminted[breaklines]{c++}{graph/spfa.cc}

\twocolumn  % 分页显示
\newpage
\section{String}
\subsection{KMP}
\inputminted[breaklines]{c++}{string/kmp.cc}

\subsection{Suffix Automaton}
\inputminted[breaklines]{c++}{string/suffix-automaton.cc}

%\newpage
%\section{Others}

\end{document}
