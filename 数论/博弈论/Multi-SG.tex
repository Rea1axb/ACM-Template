\paragraph{Multi-SG}~{}
\\
当一个单一游戏的后继为多个游戏,我们将其称为Multi-SG游戏\\
比如将一个石子堆取走一个石子后变为两个石子堆\\
因为拆分的局面和原局面在游戏图中是拓扑关系,符合SG定理\\
所以拆分局面SG值$xor$值集合$mex$函数值就是当前局面的SG值\\
~\\
~\\
\paragraph{Crosses and Crosses}~{}
\\
两个人轮流在一个$1*n$的方格纸上下棋,下的棋相同,谁先连三子谁赢\\
$1 \leq n \leq 2000$\\
~\\
为了避免让对手实现三子相连而取胜,每次落子的位置必须要和已有棋子的位置距离至少为2\\
每次落子相当于对$1*n$的棋盘进行了划分,产生一个$1*(x-3)$的棋盘和一个$1*(n-x-2)$的棋盘\\
一个长度为$n$的棋盘,其SG函数可以通过SG定理算出$SG(n)=mex(SG(i-3) \bigoplus SG(n-i-2))$\\
~\\
~\\
\paragraph{Cutting Game}~{}
\\
给出一张$n*m$的纸,两人轮流操作,每次可以在一张纸上面切一刀将其分为两半,谁先切出$1*1$的小纸片谁赢\\
$1 \leq n,m \leq 200$\\
~\\
如果有人切出了$(1,x)$或$(x,1)$,则一定是必败态,所以可以以$x$和$y$其中一个为1作为递归边界处理\\