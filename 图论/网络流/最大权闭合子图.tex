\paragraph{定义}~{}
\\
\textbf{闭合子图:}对于一个有向无环图,它的一个闭合子图满足对于其中的任意一个点,从它出发能够到达的所有点都和它在同一个闭合子图中。\\
\textbf{最大权闭合子图:}对于一个点带正/负权的有向无环图,其最大权闭合子图在所有的闭合子图中,点权和最大。

\paragraph{构图}~{}
\\
\begin{itemize}
\item 增加源点$s$和汇点$t$
\item 源$s$连接原图的正权点,容量为相应点权
\item 原图的负权点连接汇$t$,容量为相应点权的相反数
\item 原图边的容量为正无限
\end{itemize}

\paragraph{求解}~{}
\\
最大权闭合子图的点权和为:正权点和减去最小割。\\
如果割掉连接$s$的边,就是损失掉一部分收益,可以看成损失。\\
如果割掉连接$t$的边,就是付出一定代价,可以看成损失。\\
因为不会割掉$INF$的边,所以能解决选$A$必须选$B$的问题。\\
求方案时只需将$S$中的点选取即可,即从$s$出发,沿有残余流的边跑$bfs$,能够到达的点均为闭合子图中的点。

\paragraph{模型}~{}
\\
选择一些点有收益和损失,并且某些点依赖另外一些点(只有当依赖的点都存在时才能选择该点)。\\