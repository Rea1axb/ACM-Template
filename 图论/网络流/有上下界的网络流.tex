\paragraph{预备知识}~{}
\\
$f(u, v)$表示$(u, v)$的实际流量\\
$b(u, v)$表示$(u, v)$的流量下界\\
$c(u, v)$表示$(u, v)$的流量上界\\
$d[i]$表示流入$i$点的所有边的下界减去流出$i$点的所有边的下界的和\\

\paragraph{无源汇可行流}~{}
\\
\textbf{建图:}将上下界的网络流图转化成普通的网络流图\\
\begin{itemize}
\item 增加附加源点$ss$和附加汇点$tt$
\item 对于原图的边$(u, v)$,连边,流量为$c(u, v)-b(u, v)$
\item 对于每一个点,如果$d[i]>0$,连边$(ss, i)$,流量为$d[i]$;如果如果$d[i]<0$,连边$(i, tt)$,流量为$-d[i]$。\\
\end{itemize}
\textbf{求解:}在新图上跑$ss$到$tt$的最大流,若新图满流,那么存在一种可行流,此时原图中每条边的流量应为新图中边的流量+边的流量下界。\\

\paragraph{有源汇可行流}~{}
\\
\textbf{建图:}在原图中添加一条边$(t, s)$,$b(t, s)=0$,$c(t, s)=INF$,即让源点和汇点也满足流量平衡条件,就改造成了无源汇的网络流,其余方法同上\\
\textbf{求解:}同无源汇可行流\\

\paragraph{有源汇最大流}~{}
\\
\textbf{建图:}同有源汇可行流\\
\textbf{求解:}在新图上跑$ss$到$tt$的最大流,若新图满流,那么一定存在一种可行流,跑$s$到$t$的最大流$flow$,答案就是$flow$。\\

\paragraph{有源汇最小流}~{}
\\
\textbf{建图:}同无源汇可行流,注意是无汇源\\
\textbf{求解:}求$ss$到$tt$的最大流$flow1$,连边$(t, s)$,$b(t, s)=0$,$c(t, s)=INF$,求$ss$到$tt$的最大流$flow2$,若$flow1+flow2$满流则存在可行流,答案为边(t, s)的实际流量,即$f(t, s)$。\\

\paragraph{有源汇费用流}~{}
\\
\textbf{建图:}将上下界的网络流图转化成普通的网络流图\\
\begin{itemize}
\item 增加附加源点$ss$和附加汇点$tt$
\item 对于原图的边$(u, v)$,连边,流量为$c(u, v)-b(u, v)$,费用不变
\item 对于每一个点,如果$d[i]>0$,连边$(ss, i)$,流量为$d[i]$,费用为$0$;如果如果$d[i]<0$,连边$(i, tt)$,流量为$-d[i]$,费用为$0$。\\
\item 连边$(t, s)$,$b(t, s)=0$,$c(t, s)=INF$,费用为$0$。\\
\end{itemize}
\textbf{求解:}跑$ss$到$tt$的最小费用最大流,答案即为(求出的费用+原图中边的下界*边的费用)\\
注意:求出的答案是满足流量限制条件和流量平衡条件的情况下的最小费用流,
而不是在满足流量限制条件和流量平衡条件并且满足最大流的情况下的最小费用流。\\