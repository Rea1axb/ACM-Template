\paragraph{一般图}~{}
\\
\begin{itemize}
\item 对于不存在孤立点的图,|最大匹配|+|最小边覆盖|=|E|
\item |最大独立集|+|最小顶点覆盖|=|V|
\end{itemize}

\paragraph{二分图}~{}
\\
|最大匹配|=|最小顶点覆盖|

\paragraph{DAG最小路径覆盖}~{}
\\
有向图中,能经过所有点的最小路径(不是边)数\\
把原图每个点$i$拆成$x_i$和$y_i$构建二分图,
若原图存在有向边$(i, j)$,则在二分图中连边$(x_i, y_i)$,
原图最小路径覆盖即为|原图的顶点|-|二分图最大匹配|\\
方案:一条答案路径中的每条边都是二分图中匹配的$X$部指向$Y$部的一条边,所以$My[i]==-1$时,$i$为一条路径的起点,$Mx[i]==-1$时,$i$为一条路径的终点,因此对每个$My[i]==-1$的点作为起点递归查找就能找到答案路径。

\paragraph{无向图最小路径覆盖}~{}
\\
无向图中,能经过所有点的最小路径(不是边)数\\
把原图每个点$i$拆成$x_i$和$y_i$构建二分图,
若原图存在双向边$(i, j)$,则在二分图中连边$(x_i, y_i)$和$(y_i, x_i)$,
原图最小路径覆盖即为|原图的顶点|-|二分图最大匹配|/2