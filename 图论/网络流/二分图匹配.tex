\paragraph{一般图}~{}
\\
\begin{itemize}
\item 对于不存在孤立点的图,|最大匹配|+|最小边覆盖|=|E|
\item |最大独立集|+|最小顶点覆盖|=|V|
\end{itemize}

\paragraph{二分图}~{}
\\
|最大匹配|=|最小顶点覆盖|
\paragraph{霍尔定理}~{}
\\
首先对于二分图 $G=(X \cup Y, E),$ 点集被分为了 $X$ 和 $Y$ 两部分。 是否具有完美匹配,首先一个最基本的条件就是 $|X|=|Y|_{\circ} \quad$ Hall 定理则在此基础上给出了一个更强的条件。 对于一个点集 $T \subseteq X,$ 定义 $\Gamma(T)$ 如下:$\Gamma(T)=\{v \mid u \rightarrow v \in E, u \in T, v \in Y\}$,即表示$T$中所有点能够直接到达的$Y$中的点的集合。Hall 条件用于判断一个二分图是否存在完美匹配。如果对于任意的点集 $T \subseteq X,$ 均存 在:
$$
|T| \leq|\Gamma(T)|
$$
称此二分图满足 Hall 条件。\\
Hall 定理的表述如下:
二分图 $G(X \cup Y, E)$ 存在完美匹配当且仅当 $|X|=|Y|$ 并且满足 Hall 条件。

\paragraph{正则二分图}~{}
\\
如果一个二分图中每个点的度数都是 $k,$ 那么称作一个 $k$ -正则二分图。
显然一个正则二分图左右两边点数相同 (因为边数 $=k|L|=k|R|$ ) 。下面以 $N$ 指代 $|L|$ 或 $|R|$ 。
如果 $S \subseteq L,$ 那么连到 $S$ 的边有 $k|S|$ 条, 连到 $N(S)$ 的边有 $k|N(S)|$ 条。由于前者是后者的子集, 我们就有 $|S| \leq$ $|N(S)|$。
所以正则二分图满足 Hall 定理的条件, 其必定存在完美匹配。
\paragraph{$k=2^{d}$ 时的完美匹配}~{}
\\
如果给定一个 $2^{d}$ -正则二分图, 如何找出其一个完美匹配?
这个算法很容易: 直接找出一条欧拉回路, 这样就给所有边定了向; 且每个点出度入度相同。删掉某一个方向的所有边, 然后忽略掉定向, 就变成了 $2^{d-1}$ -正则二分图。递归直到 $d=0$ 即可。
复杂度为 $2^{d} N+2^{d-1} N+\cdots+N=O\left(2^{d} N\right),$ 也就是 $O(m)$。
\paragraph{$k$任意时的完美匹配}~{}
\\
这是一个随机算法,随机在运行时间上。
算法如下:\\
- 重复 $N$ 次:\\
- 随机选一个左边的未匹配点, 然后沿增广路随机游走(即从左往右随机走未匹配边,从右往左走匹配边),直到走到一
个右边的未匹配点。\\
- 把走出来的环去掉(找到最后一个出现过多次的点, 然后把第一次走到它到最后一次走到它中间的这段路砍掉) 。这样
就找到了一条增广路。对它进行增广以把匹配数增加 1 。\\
可以证明:当还剩下 $2 T$ 个未匹配点 (即重复过 $N-T$ 次) 时, 随机游走的步数期望是
$$
2 \frac{(N-T)(k-1)}{k T}+1
$$
这个东西直接放缩成 $\frac{2 N}{T},$ 所以期望复杂度是
$$
O\left(\sum_{T=1}^{N} \frac{2 N}{T}\right)=O(N \log N)
$$
\paragraph{DAG最小不相交路径覆盖}~{}
\\
有向图中,能经过所有点的最小不相交路径(不是边)数\\
把原图每个点$i$拆成$x_i$和$y_i$构建二分图,
若原图存在有向边$(i, j)$,则在二分图中连边$(x_i, y_i)$,
原图最小路径覆盖即为|原图的顶点|-|二分图最大匹配|。\\
方案:一条答案路径中的每条边都是二分图中匹配的$X$部指向$Y$部的一条边,所以$My[i]==-1$时,$i$为一条路径的起点,$Mx[i]==-1$时,$i$为一条路径的终点,因此对每个$My[i]==-1$的点作为起点递归查找就能找到答案路径。
\paragraph{DAG最小可相交路径覆盖}~{}
\\
先用$floyd$求出原图的传递闭包,即如果$a$到$b$有路径,那么就加边$(a, b)$。然后就转化成了最小不相交路径覆盖问题。
