\paragraph{说明}~{}
\\
通常是用一些方式将问题转化成上述四种模型
\paragraph{例题1}~{
    \\
    \textbf{题意:}用黑白两种颜色对一个$N \times M$的矩阵中的元素进行染色,矩阵中每个元素被染色都会获得相应的收益:$B[i][j]$为染成黑色的收益,$W[i][j]$为染成白色的收益。并且如果相邻两个元素的颜色相同会有额外收益。问最大收益
    \\
    \textbf{题解:}$s$向每个点连边,流量为$B[i][j]$,表示该点染成黑色的收益;每个点向$t$连边,流量为$W[i][j]$,表示该点染成白色的收益;对于每对相邻的点,新建一个节点$x$,向两个点分别连边,流量均为$INF$,$s$向$x$点连边,流量为相同颜色的收益,表示将两个点染成黑色的收益,新建一个节点$y$,两个点分别向$y$连边,流量均为$INF$,$y$向$t$点连边,流量为相同颜色的收益,表示将两个点染成白色的收益。答案为:收益和-最小割。
}
\paragraph{例题2}~{
    \\
    \textbf{题意:}一个$N \times M$的矩阵中,有些元素已经有黑色或白色,有些没有颜色,对没有颜色的元素进行染色,如果相邻两个元素的颜色不同会有收益。问最大收益\\
    \textbf{题解:}因为收益只会在相邻元素之间产生,而相邻元素的横纵坐标之和的奇偶性不同所以可以分成二分图,因此,将二分图中一边的点的颜色反过来,那么收益就是相邻相同颜色的点对产生,对于已经染色的点,其对答案的贡献可直接算出来,只需考虑没有颜色的点,遍历四个方向就能算出该点染成黑色和染成白色的收益,而如果有相邻的没有颜色的点对,就相当于染成相同颜色有额外收益。那么该问题就转化成例题1了,最终答案为:已染色收益+未染色收益和-最小割。 
}
