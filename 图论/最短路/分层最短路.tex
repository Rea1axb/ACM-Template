\paragraph{一般模型}~{}
\\
在一个正常的图上可以进行k次决策,对于每次决策,不影响图的结构,只影响目前的状态或代价。一般将决策前的状态和决策后的状态之间连接一条权值为决策代价的边,表示付出该代价后就可以转换状态了。\\
\paragraph{例题1}~{}
\\
基本最短路问题,条件加上可以使得边权花费变$0$的$k$次机会。
\paragraph{题解}~{}
\\
将原图扩成$k+1$层,第$j$层表示用了$j$次机会,$dist[x][k]$表示终点为$x$,用了$k$次机会的最小花费。\\
\paragraph{例题2}~{}
\\
求起点到点$x$的路径中$\sum\limits_{i=1}^{k}{w_{e_i}} - \max\limits_{i=1}^{k}{w_{e_i}} + \min\limits_{i=1}^{k}{w_{e_i}}$的最小值。
\paragraph{题解}~{}
\\
对最短路问题增加两种决策:\\
1.使某一条边花费变为$0$,准确来说是减少$w_{e_i}$。\\
2.使某一条边花费增加$w_{e_i}$。\\
两种决策都必须使用一次。与例题1同样解法,$dist[x][i][j]$表示终点为$x$,用了$i$次第一种决策,$j$次第二种决策的最小花费。\\
可通过贪心性质证明$dist[x][1][1]$就是原问题的答案。