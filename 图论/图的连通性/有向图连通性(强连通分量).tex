\paragraph{SCC相关性质}~{}
\\
\textbf{一张有向图至少要加多少条边才能使整张图为强连通分量?}\\
若原本就是强连通分量,答案为$0$。\\
否则,将所有强连通分量缩点后,令入度为$0$的点的数量为$cnt1$,出度为$0$的点的数量为$cnt2$,答案为$\max(cnt1, cnt2)$。\\
\textbf{简易证明:}\\
强连通分量中的点的入度和出度都大于$0$,一条边能带来$1$个入度和$1$个出度。\\

\paragraph{仙人掌图}~{}
\\
\textbf{有向图定义:}\\
1.必须是一个强连通图。\\
2.每条边只能属于一个环。\\
\textbf{有向图性质(用于判断是否是有向仙人掌图):}\\
1.仙人掌图的$DFS$树没有横向边。\\
2.$low[v] \leq dfn[u]$。\\
3.设点$u$有$cnt1[u]$个儿子的$low$值小于$dfn[u]$,同时$u$有$cnt2[u]$条前向边,那么$cnt1[u]+cnt2[u]<2$。\\
\textbf{无向图定义(用于判断是否是无向仙人掌图):}\\
1.必须是一个连通图。\\
2.每条边至多属于一个环。\\
\textbf{如何判断一条边属于几个环?}\\
在$dfs$树中,对于一条成环边$(u, v)$,将$u$和$v$路径上的边的值$+1$,树上边差分维护即可。