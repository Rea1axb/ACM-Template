\paragraph{矩阵在图论中的应用}~{}
\\
$G^{K}[a][b]$:$a$到$b$路径长度为K的路径数,若将矩阵乘法修改为$C[i][j] |= A[i][k]\&B[k][j]$,则表示$a$到$b$是否存在路径长度为$K$的路径。\\
\paragraph{求经过K条边的最短路径}~{}
\\
修改矩阵乘法为:$C[i][j]=\min_{k=1}^{m}(A[i][k]+B[k][j])$\\
若$i$,	$j$有边相连,则$G[i][j]$为边权,否则为$INF$\\
\paragraph{两点间通过边数恰好为K的路径中,最大边的最小值}~{}
\\
修改矩阵乘法为:$C[i][j]=\min_{k=1}^{m}(\max(A[i][k], B[k][j]))$\\
若$i$,	$j$有边相连,则$G[i][j]$为边权,否则为$INF$\\
\paragraph{两点间通过边数不超过K的路径中,最大边的最小值}~{}
\\
把每个点$i$都增加一个权值为$-INF$的自环,之后做法和上题一样